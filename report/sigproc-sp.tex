% THIS IS SIGPROC-SP.TEX - VERSION 3.1
% WORKS WITH V3.2SP OF ACM_PROC_ARTICLE-SP.CLS
% APRIL 2009
%
% It is an example file showing how to use the 'acm_proc_article-sp.cls' V3.2SP
% LaTeX2e document class file for Conference Proceedings submissions.
% ----------------------------------------------------------------------------------------------------------------
% This .tex file (and associated .cls V3.2SP) *DOES NOT* produce:
%       1) The Permission Statement
%       2) The Conference (location) Info information
%       3) The Copyright Line with ACM data
%       4) Page numbering
% ---------------------------------------------------------------------------------------------------------------
% It is an example which *does* use the .bib file (from which the .bbl file
% is produced).
% REMEMBER HOWEVER: After having produced the .bbl file,
% and prior to final submission,
% you need to 'insert'  your .bbl file into your source .tex file so as to provide
% ONE 'self-contained' source file.
%
% Questions regarding SIGS should be sent to
% Adrienne Griscti ---> griscti@acm.org
%
% Questions/suggestions regarding the guidelines, .tex and .cls files, etc. to
% Gerald Murray ---> murray@hq.acm.org
%
% For tracking purposes - this is V3.1SP - APRIL 2009

\documentclass{acm_proc_article-sp}

\usepackage{listings}

\begin{document}

\title{Summarization: data analysis on italian articles}

% You need the command \numberofauthors to handle the 'placement
% and alignment' of the authors beneath the title.
%
% For aesthetic reasons, we recommend 'three authors at a time'
% i.e. three 'name/affiliation blocks' be placed beneath the title.
%
% NOTE: You are NOT restricted in how many 'rows' of
% "name/affiliations" may appear. We just ask that you restrict
% the number of 'columns' to three.
%
% Because of the available 'opening page real-estate'
% we ask you to refrain from putting more than six authors
% (two rows with three columns) beneath the article title.
% More than six makes the first-page appear very cluttered indeed.
%
% Use the \alignauthor commands to handle the names
% and affiliations for an 'aesthetic maximum' of six authors.
% Add names, affiliations, addresses for
% the seventh etc. author(s) as the argument for the
% \additionalauthors command.
% These 'additional authors' will be output/set for you
% without further effort on your part as the last section in
% the body of your article BEFORE References or any Appendices.

\numberofauthors{2} %  in this sample file, there are a *total*
% of EIGHT authors. SIX appear on the 'first-page' (for formatting
% reasons) and the remaining two appear in the \additionalauthors section.
%
\author{
% You can go ahead and credit any number of authors here,
% e.g. one 'row of three' or two rows (consisting of one row of three
% and a second row of one, two or three).
%
% The command \alignauthor (no curly braces needed) should
% precede each author name, affiliation/snail-mail address and
% e-mail address. Additionally, tag each line of
% affiliation/address with \affaddr, and tag the
% e-mail address with \email.
%
% 1st. author
\alignauthor
       Roberto Zen - 172181\\
       \vspace{1mm}
       \affaddr{University of Trento}\\
       \vspace{1mm}
       \email{roberto.zen@studenti.unitn.it}
% 2nd. author
\alignauthor
       Luca Zamboni - 175606\\
       \vspace{1mm}
       \affaddr{University of Trento}\\
       \vspace{1mm}
       \email{luca.zamboni@studenti.unitn.it}
}


% – Introduction: An introduction to the problem. Why it is interesting? Why would anyone care? Why is your solution good?
% – Motivating Example: An example that shows what you are trying to do.
% - Problem Statement: A problem definition. What problem exactly you try to solve
% – Solution: A description of the solution you provide and its implementation.
% - Related work: A description of other related approaches you have seen in the literature
% - Experimental Evaluation: Multiple experiments that show that your approach works well  comparison  with related work, stress-test to higher data sizes, user evaluation etc.) 

\maketitle
\begin{abstract}
\vspace{2mm}
This is a beautiful abstract.
\end{abstract}

\section{Introduction}
\vspace{2mm}
Every day lots of articles are published by many newspapers and websites. As a consequence, the same news is repeated in many sources causing confusion to readers which have to choose the most interesting ones. Fortunately, there are websites which everyday explore articles from different surces and group them together. The most famous one is Google News\footnote{https://news.google.com}. It actually \textit{aggregates} headlines from news sources worldwide which treat the same news and gives to readers a list of the most \textit{representative} ones for each group. While the title provided by Google is the same as the one written in the source, the body is a paragraph of the article which belongs to the source and allows readers to get instantaneously significant information. The aims of our work is to replicate what Google News provides, focusing on italian articles. The proposed solution provides to readers a list of articles with no duplicates: for each group of similar news we provide the most representative one to show.

\section{Motivating Example}
\vspace{2mm}
The best example from which we took inspiration is Google News. It is simple, fast and personilizable. News may be filtered by personalized interests, locations and other preferences. The aim of our work is to help readers to read news that really represent their interests. In this way, readers do not have to read all the articles that are published and do not have to skip news for which they are not interested in.

%\section{Problem Statement}
%vspace{2mm}
%The problem we face in is twofold: a tipical scenario as the one described in the section 1 can be divided into two steps. The first phase is to find the right way to aggregate similar articles. The second phase is to chose the article from each group which represent the entire cluster. The first challenge we face in was to understand how when different news have to be considered similar.

\section{Data mining}
\vspace{2mm}
\subsection{The Problem}
\vspace{2mm}
The problem is to present to the user tha main news happened the day before. We start from a dataset of news collected from different source. Our goals is divided in two main task:
\begin{enumerate}
\item Find similar news in a day and cluster it together.
\item Find a rappresentative news for each cluster to present to the user.
\end{enumerate}
We also try to have a trade of beetween quality and performance. We need high performance also because the idea is replicate google news that has a lot of sources for each country and a lot of users with gigabytes of data.
\subsection{Solution}
The first step is remove eventual news duplication. This is caused because different newspapers copy the text of the news from other newspapers.\\
The second step is to remove the stop word from the dataset. We remove stop word because they are common and repeated for every news so they doesn't give any information about the news.\\
For similarity measure of the news we need to have an high value if the two news have a lot of common word over the totality of the word of each news.For this fact we choosed as similarity the jaccard similarity. The jaccard similarity is largerly used for comparing sets and document like our case. The jaccard similarity is defined in the following formula.$$Jaccard(A,B) = \frac{\left\vert A \cap B \right\vert}{\left\vert A \cup B \right\vert}$$
To optimize the jaccard similarity we have builded a matrix where the row are the element present in the all set and the column are the document. The cell can take value 1 if the element is present in the relative document and 0 if not. With this matrix every jaccard similarity computation have a complexity on O(n) and in this way we compare number which cost a lot less of comparing string. \\
With this matrix we can do another stop of optimization. We can build a signature matrix using a min hash function as described in [CIT]. If the number of permutation is high (e.g. 100 permutation) this signature matrix preserve the distance beetween document loosing only a few information. The advantage of having this signature matrix is that it occupies in memory a space proportion to (Documents per Number of permutation) instead of having (Document per Number of item). In this way we save a lot space and a lot time in computation.

We tryed different tecniques to try to cluster news.
Our first attempt to cluster news is the agglomerative clustering. This is the very basic idea. We start with clusters of one news each. Now with a recoursive strategy we continuesly merge the 2 cluster with the lowest distance. We stop we the distance of the two closer cluster over a threashold value.\\
The distance measure used is: $$ 1 - jaccard(C_1,C_2)$$
We always merge the two closer cluster. To determine the distance beetween to cluster we used different tecnique. The first one is to choose for each of the two cluster the closest point to the other cluster and take the distance of this two point as the distance of the two cluster. The second tecnique, similarly as the first, is to choose for each of the two cluster the further point to the other cluster and take the distance of this two point as the distance of the two cluster. The third is take the average of the distance beetween the point of the two cluster as the distance of the two clusters.\\
We this clustering method we faced 2 main problems. The first one is that there was always few cluster really big with a lot of news and a lot of small clusters. This happened with all of the three of distance measure beetween clusters, also with takeing the average that we taught it was the best. The second problem is the difficulty of determine the threshold value to stop the clustering. If is too high there are few clusters with a lot of news and if it is to small there are a too much clusters with few news each.\\
After we faced this two problems we decided to abandon this idea and we changed clustering method.

The second method used.

\section{Big Data}
\vspace{2mm}

\subsection{The dataset}
\vspace{2mm}
We started by collecting data from the Google News's website. We built a crawler which downloads the feed rss of all the news which belong to the home page. 
Each feed rss downloaded by the crawler is parsed and for each news written in it we extract some attributes. The most relevant ones are the following: title, body, name of the source, published date, url of the feed, url of the source.
Afterward, the news are stored in a json file. We decided to store articles in a single file rather than divide them in multiple files. However, this process can be easily converted in a distributed one using a NoSQL Database Management System like MongoDb, in which each document is an article.

\subsection{Text optimization with Spark}
\vspace{2mm}
Once we stored the list of news, we use Spark to parallelize some functions.

% REMOVE STOP WORDS
% l = jsc.parallelize(fromNewsToTuple(list_news))
% l = l.map(lambda n:(n[0],remove_stop_words_from_string(n[1],stop_words),remove_stop_words_from_string(n[2],stop_words))).collect()
% list_news = reassemblyNews(list_news,l)

Firstly, a combination of map-reduce Spark's APIs is used in order to remove stop words from both the title and the body of each article. In fact, the operation of removing words from text is independent from one article to another and it has been implemented as follows: first, the list of news is mapped into a list of tuples of the form \textit{(id, title, body)} so the remove function can be done in parallel by working on single articles. Afterwards, the remove of the stop words is performed. As result a new tuple is provided with no stop words, so the entire list of tuples can be merged into a list of news overriding the old values of title and body with the new ones.

% KEYWORDS

% jsc = SparkContext(appName="LOADNEWS: Get keywords")
% l = jsc.parallelize(formNewsToIdTitleBodyUrl(list_news))
% l = l.map(lambda n:(n[0], get_keywords((n[1],n[2],n[3])))).collect()
% list_news = reassemblyNewsAndSetKeywords(list_news,l)
% jsc.stop()

Secondly, the keywords and entities extraction operation is performed in Spark as follows: given the list of news, a map function is used to split that list, then each news is treated independently so the keywords that belong to the title, the body and the feed url can be extract in parallel. As result of the previous operation, the set of keywords for each article is taken as input by the merge function which stores them in a new attribute.

Thirdly, duplicates are also removed using Spark. In this case, the map function works as follows: news are considered as a tuple \textit{(id,title)} so 

% \lstset{language=Python} 
% \begin{lstlisting}
% l = jsc.parallelize(formNewsToTuple(list_news))
% tuple = (n[0], get_keywords((n[1],n[2],n[3])))
% l = l.map(lambda n:(tuple))
% l.collect()
% list_news = reassemblyNews(list_news,l)
% \end{lstlisting}




\subsection{Math Equations}
You may want to display math equations in three distinct styles:
inline, numbered or non-numbered display.  Each of
the three are discussed in the next sections.

\subsubsection{Inline (In-text) Equations}
A formula that appears in the running text is called an
inline or in-text formula.  It is produced by the
\textbf{math} environment, which can be
invoked with the usual \texttt{{\char'134}begin. . .{\char'134}end}
construction or with the short form \texttt{\$. . .\$}. You
can use any of the symbols and structures,
from $\alpha$ to $\omega$, available in
\LaTeX this section will simply show a
few examples of in-text equations in context. Notice how
this equation: \begin{math}\lim_{n\rightarrow \infty}x=0\end{math},
set here in in-line math style, looks slightly different when
set in display style.  (See next section).

\subsubsection{Display Equations}
A numbered display equation -- one set off by vertical space
from the text and centered horizontally -- is produced
by the \textbf{equation} environment. An unnumbered display
equation is produced by the \textbf{displaymath} environment.

Again, in either environment, you can use any of the symbols
and structures available in \LaTeX; this section will just
give a couple of examples of display equations in context.
First, consider the equation, shown as an inline equation above:
\begin{equation}\lim_{n\rightarrow \infty}x=0\end{equation}
Notice how it is formatted somewhat differently in
the \textbf{displaymath}
environment.  Now, we'll enter an unnumbered equation:
\begin{displaymath}\sum_{i=0}^{\infty} x + 1\end{displaymath}
and follow it with another numbered equation:
\begin{equation}\sum_{i=0}^{\infty}x_i=\int_{0}^{\pi+2} f\end{equation}
just to demonstrate \LaTeX's able handling of numbering.

\subsection{Citations}
Citations to articles,
conference
proceedings listed
in the Bibliography section of your
article will occur throughout the text of your article.
You should use BibTeX to automatically produce this bibliography;
you simply need to insert one of several citation commands with
a key of the item cited in the proper location in
the \texttt{.tex} file.
The key is a short reference you invent to uniquely
identify each work; in this sample document, the key is
the first author's surname and a
word from the title.  This identifying key is included
with each item in the \texttt{.bib} file for your article.

The details of the construction of the \texttt{.bib} file
are beyond the scope of this sample document, but more
information can be found in the \textit{Author's Guide},
and exhaustive details in the \textit{\LaTeX\ User's
Guide}.

This article shows only the plainest form
of the citation command, using \texttt{{\char'134}cite}.
This is what is stipulated in the SIGS style specifications.
No other citation format is endorsed.

\subsection{Tables}
Because tables cannot be split across pages, the best
placement for them is typically the top of the page
nearest their initial cite.  To
ensure this proper ``floating'' placement of tables, use the
environment \textbf{table} to enclose the table's contents and
the table caption.  The contents of the table itself must go
in the \textbf{tabular} environment, to
be aligned properly in rows and columns, with the desired
horizontal and vertical rules.  Again, detailed instructions
on \textbf{tabular} material
is found in the \textit{\LaTeX\ User's Guide}.

Immediately following this sentence is the point at which
Table 1 is included in the input file; compare the
placement of the table here with the table in the printed
dvi output of this document.

\begin{table}
\centering
\caption{Frequency of Special Characters}
\begin{tabular}{|c|c|l|} \hline
Non-English or Math&Frequency&Comments\\ \hline
\O & 1 in 1,000& For Swedish names\\ \hline
$\pi$ & 1 in 5& Common in math\\ \hline
\$ & 4 in 5 & Used in business\\ \hline
$\Psi^2_1$ & 1 in 40,000& Unexplained usage\\
\hline\end{tabular}
\end{table}

To set a wider table, which takes up the whole width of
the page's live area, use the environment
\textbf{table*} to enclose the table's contents and
the table caption.  As with a single-column table, this wide
table will ``float" to a location deemed more desirable.
Immediately following this sentence is the point at which
Table 2 is included in the input file; again, it is
instructive to compare the placement of the
table here with the table in the printed dvi
output of this document.


\begin{table*}
\centering
\caption{Some Typical Commands}
\begin{tabular}{|c|c|l|} \hline
Command&A Number&Comments\\ \hline
\texttt{{\char'134}alignauthor} & 100& Author alignment\\ \hline
\texttt{{\char'134}numberofauthors}& 200& Author enumeration\\ \hline
\texttt{{\char'134}table}& 300 & For tables\\ \hline
\texttt{{\char'134}table*}& 400& For wider tables\\ \hline\end{tabular}
\end{table*}
% end the environment with {table*}, NOTE not {table}!

\subsection{Figures}
Like tables, figures cannot be split across pages; the
best placement for them
is typically the top or the bottom of the page nearest
their initial cite.  To ensure this proper ``floating'' placement
of figures, use the environment
\textbf{figure} to enclose the figure and its caption.

This sample document contains examples of \textbf{.eps}
and \textbf{.ps} files to be displayable with \LaTeX.  More
details on each of these is found in the \textit{Author's Guide}.

As was the case with tables, you may want a figure
that spans two columns.  To do this, and still to
ensure proper ``floating'' placement of tables, use the environment
\textbf{figure*} to enclose the figure and its caption.

Note that either {\textbf{.ps}} or {\textbf{.eps}} formats are
used; use
the \texttt{{\char'134}epsfig} or \texttt{{\char'134}psfig}
commands as appropriate for the different file types.

\subsection{Theorem-like Constructs}
Other common constructs that may occur in your article are
the forms for logical constructs like theorems, axioms,
corollaries and proofs.  There are
two forms, one produced by the
command \texttt{{\char'134}newtheorem} and the
other by the command \texttt{{\char'134}newdef}; perhaps
the clearest and easiest way to distinguish them is
to compare the two in the output of this sample document:

This uses the \textbf{theorem} environment, created by
the\linebreak\texttt{{\char'134}newtheorem} command:
\newtheorem{theorem}{Theorem}
\begin{theorem}
Let $f$ be continuous on $[a,b]$.  If $G$ is
an antiderivative for $f$ on $[a,b]$, then
\begin{displaymath}\int^b_af(t)dt = G(b) - G(a).\end{displaymath}
\end{theorem}

The other uses the \textbf{definition} environment, created
by the \texttt{{\char'134}newdef} command:
\newdef{definition}{Definition}
\begin{definition}
If $z$ is irrational, then by $e^z$ we mean the
unique number which has
logarithm $z$: \begin{displaymath}{\log e^z = z}\end{displaymath}
\end{definition}

Two lists of constructs that use one of these
forms is given in the
\textit{Author's  Guidelines}.
and don't forget to end the environment with
{figure*}, not {figure}!
 
There is one other similar construct environment, which is
already set up
for you; i.e. you must \textit{not} use
a \texttt{{\char'134}newdef} command to
create it: the \textbf{proof} environment.  Here
is a example of its use:
\begin{proof}
Suppose on the contrary there exists a real number $L$ such that
\begin{displaymath}
\lim_{x\rightarrow\infty} \frac{f(x)}{g(x)} = L.
\end{displaymath}
Then
\begin{displaymath}
l=\lim_{x\rightarrow c} f(x)
= \lim_{x\rightarrow c}
\left[ g{x} \cdot \frac{f(x)}{g(x)} \right ]
= \lim_{x\rightarrow c} g(x) \cdot \lim_{x\rightarrow c}
\frac{f(x)}{g(x)} = 0\cdot L = 0,
\end{displaymath}
which contradicts our assumption that $l\neq 0$.
\end{proof}

Complete rules about using these environments and using the
two different creation commands are in the
\textit{Author's Guide}; please consult it for more
detailed instructions.  If you need to use another construct,
not listed therein, which you want to have the same
formatting as the Theorem
or the Definition shown above,
use the \texttt{{\char'134}newtheorem} or the
\texttt{{\char'134}newdef} command,
respectively, to create it.

\subsection*{A {\secit Caveat} for the \TeX\ Expert}
Because you have just been given permission to
use the \texttt{{\char'134}newdef} command to create a
new form, you might think you can
use \TeX's \texttt{{\char'134}def} to create a
new command: \textit{Please refrain from doing this!}
Remember that your \LaTeX\ source code is primarily intended
to create camera-ready copy, but may be converted
to other forms -- e.g. HTML. If you inadvertently omit
some or all of the \texttt{{\char'134}def}s recompilation will
be, to say the least, problematic.

\section{Conclusions}
This paragraph will end the body of this sample document.
Remember that you might still have Acknowledgments or
Appendices; brief samples of these
follow.  There is still the Bibliography to deal with; and
we will make a disclaimer about that here: with the exception
of the reference to the \LaTeX\ book, the citations in
this paper are to articles which have nothing to
do with the present subject and are used as
examples only.
%\end{document}  % This is where a 'short' article might terminate

%ACKNOWLEDGMENTS are optional
\section{Acknowledgments}
This section is optional; it is a location for you
to acknowledge grants, funding, editing assistance and
what have you.  In the present case, for example, the
authors would like to thank Gerald Murray of ACM for
his help in codifying this \textit{Author's Guide}
and the \textbf{.cls} and \textbf{.tex} files that it describes.

%
% The following two commands are all you need in the
% initial runs of your .tex file to
% produce the bibliography for the citations in your paper.
\bibliographystyle{abbrv}
\bibliography{sigproc}  % sigproc.bib is the name of the Bibliography in this case
% You must have a proper ".bib" file
%  and remember to run:
% latex bibtex latex latex
% to resolve all references
%
% ACM needs 'a single self-contained file'!
%
%APPENDICES are optional
%\balancecolumns
% \appendix
%Appendix A

% \section{Headings in Appendices}
% The rules about hierarchical headings discussed above for
% the body of the article are different in the appendices.
% In the \textbf{appendix} environment, the command
% \textbf{section} is used to
% indicate the start of each Appendix, with alphabetic order
% designation (i.e. the first is A, the second B, etc.) and
% a title (if you include one).  So, if you need
% hierarchical structure
% \textit{within} an Appendix, start with \textbf{subsection} as the
% highest level. Here is an outline of the body of this
% document in Appendix-appropriate form:

% \subsection{Introduction}
% \subsection{The Body of the Paper}
% \subsubsection{Type Changes and  Special Characters}
% \subsubsection{Math Equations}
% \paragraph{Inline (In-text) Equations}
% \paragraph{Display Equations}
% \subsubsection{Citations}
% \subsubsection{Tables}
% \subsubsection{Figures}
% \subsubsection{Theorem-like Constructs}
% \subsubsection*{A Caveat for the \TeX\ Expert}
% \subsection{Conclusions}
% \subsection{Acknowledgments}
% \subsection{Additional Authors}

\subsection{References}
Generated by bibtex from your ~.bib file.  Run latex,
then bibtex, then latex twice (to resolve references)
to create the ~.bbl file.  Insert that ~.bbl file into
the .tex source file and comment out
the command \texttt{{\char'134}thebibliography}.
% This next section command marks the start of
% Appendix B, and does not continue the present hierarchy
\balancecolumns
% That's all folks!
\end{document}
